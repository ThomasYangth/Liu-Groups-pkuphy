\documentclass{beamer}
\usepackage{amsmath}
\usepackage{xeCJK}
\usepackage{graphicx}
\usefonttheme[onlymath]{serif}
\setCJKmainfont[BoldFont={STZhongsong},ItalicFont={FangSong}]{SimSun}
\setCJKsansfont[BoldFont={DengXian Bold},ItalicFont={KaiTi}]{DengXian}
\newcommand{\pdfpage}[1]{{\usebackgroundtemplate{\includegraphics[width=\paperwidth]{W2/W2_页面_#1.jpg}}\begin{frame}[plain]\label{OrigPdf#1}\end{frame}}}%将原课件的某一页作为一个全页图片插入,参数为原课件页数(图片用pdf的导出图像功能得到),注意个位数需要前面加一个零
\newcommand{\refpage}[1]{注第\ref{OrigPdf#1}页}%将原页码转化为新页码
\newcommand{\refpageN}[1]{\ref{OrigPdf#1}}%将原页码转化为新页码,只给出数字
\newcommand{\pp}[2]{\frac{\partial #1}{\partial #2}}
\newcommand{\ppat}[3]{\left.\left(\pp{#1}{#2}\right)\right|_{#3}}
\newcommand{\di}{\mathrm d}
\title{《李群和李代数》第二周课件勘误与注解}
\author{原课件:刘玉鑫老师}
\begin{document}
\maketitle
\begin{frame}{说明}
本文档基于原课件,修改了一些其中的错误并对一些不太明晰的地方进行了注解。\\
主要参考书:
\begin{itemize}
\item P. M. Cohn, \textit{Lie Groups}
\item Brain Hall, \textit{Lie Groups, Lie Algebras and Representations}
\item A. O. Barut, \textit{Theory of Group Representations and Applications}
\item 梁灿彬,《微分几何与广义相对论》
\item V. S. Varadarajan, \textit{Lie Groups, Lie Algebras and Their Representations}
\item 韩其智、孙洪洲,《群论》
\end{itemize}


本文档的页面由三种页面组成:(1)直接引用课件;(2)引用原课件,但改写了其中的错误,表现为白底黑字蓝色标题的页面;(3)对于原课件的注解,同样是白底黑字蓝色标题的页面,但标题以“注第XX页”开头。

本文中提到的页码均指本pdf页码,不是原课件页码。

部分结果来自群内讨论。如有问题,欢迎讨论。
\end{frame}

\pdfpage{01}
\pdfpage{02}
\pdfpage{03}
\pdfpage{04}
\pdfpage{05}
\pdfpage{06}
\pdfpage{07}
\pdfpage{08}
\pdfpage{09}
\pdfpage{10}
\pdfpage{11}
\pdfpage{12}
\pdfpage{13}
\pdfpage{14}
\pdfpage{15}
\pdfpage{16}
\pdfpage{17}
\pdfpage{18}
\pdfpage{19}
\pdfpage{20}
\pdfpage{21}
\pdfpage{22}
\pdfpage{23}
\pdfpage{24}
\pdfpage{25}
\pdfpage{26}
\pdfpage{27}
\pdfpage{28}
\pdfpage{29}
\pdfpage{30}
\pdfpage{31}
\pdfpage{32}
\pdfpage{33}
\pdfpage{34}
\pdfpage{35}

\end{document}